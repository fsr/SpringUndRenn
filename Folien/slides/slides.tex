% The Slide Definitions
\input{../templates/course_definitions}

% Author and Course information
\input{../templates/course_information}

% Presentation title
% TODO Change the topic of the lesson
\title{Workshop: Livecoding eines Jump and Run}
\date{\today}


\begin{document}
\maketitle

%--------------------------------------------------------------------

\begin{frame}[fragile]{Repo klonen}
 Ein neues Verzeichnis erstellen:
 \begin{lstlisting}
$ mkdir <Name>
 \end{lstlisting}
 
 In das Verzeichnis wechseln:
 \begin{lstlisting}
$ cd <Name>
 \end{lstlisting}

 Das Repo mit Git klonen:
 \begin{lstlisting}
[<Name>]$ git clone https://github.com/fsr/SpringUndRenn.git
 \end{lstlisting}
\end{frame}

%--------------------------------------------------------------------

\begin{frame}[fragile]{Virtuelle Maschine vorbereiten}
 VM erstellen:
 \begin{lstlisting}
[<Name>]$ python -m venv <venvName>
 \end{lstlisting}
 
 VM starten:
 \begin{lstlisting}
[<Name>]$ source <venvName>/bin/activate
 \end{lstlisting}

 Pygame installieren:
 \begin{lstlisting}
(venvName)[<Name>]$ pip install pygame
 \end{lstlisting}
\end{frame}

%--------------------------------------------------------------------

\begin{frame}[fragile]{Spiel starten}
 VM starten, falls noch nicht geschehen:
 \begin{lstlisting}
[<Name>]$ source <venvName>/bin/activate
 \end{lstlisting}

 In das source Verzeichnis wechseln:
 \begin{lstlisting}
(venvName)[<Name>]$ cd SpringUndRenn/src
 \end{lstlisting}
 
  Spiel starten:
 \begin{lstlisting}
(venvName)[<src>]$ python game.py
 \end{lstlisting}
\end{frame}

%--------------------------------------------------------------------

\end{document}
